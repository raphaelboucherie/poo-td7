\documentclass[11pt,a4paper]{article}

\usepackage[utf8]{inputenc}
\usepackage[french]{babel}
\usepackage[top=3cm,bottom=2cm,left=2cm,right=2cm]{geometry}
\usepackage{fancyhdr}
\usepackage{graphicx}
\usepackage{wrapfig}

\pagestyle{fancy}
\lhead{POO - TD6}
\rhead{\'Equipe Optiplex}
\lfoot{\bsc{Enseirb-Matmeca}}
\rfoot{Informatique - I2}
\renewcommand{\footrulewidth}{0.5px}

\begin{document}

\begin{center}
\huge{\textbf{Gestion d’erreurs du paquetage tec}}
\end{center}

\begin{flushleft}
\emph{\'Equipe Optiplex \\ Yannick Levif, Joffrey Diebold, Mohamed Akdim, Jérôme Le Bot}
\end{flushleft}
\section{Gestion d’erreurs et instanciation}

Les cas limites de l'instanciation de JaugeNaturel sont rencontrés lorsque $VigieMax = VigieMin = Valeur$ ou $VigieMax < VigieMin$.\\

\begin{verbatim}
public void testExceptionCasLimites() { JaugeNaturel inverse = new JaugeNaturel(78, 13, 55); JaugeNaturel egale = new JaugeNaturel(-42, -42, -42); }
\end{verbatim}

\textit{Nous avons pris le soin de modifier le code de l'instanciation de la JaugeNaturel egale car le résultat est incohérent dans le cas donné dans l'énoncé.}

Le résultat de l'instanciation est : \\
Pour la Jauge inverse: \\
estRouge(): true
estVert(): false
estBleu(): true
L'état est donc bien incohérent.

Pour la Jauge egale:
estRouge(): true
estVert(): false
estBleu(): true
L'état est donc bien incohérent.

\subsection{Lever une exception}

On ne relève que la première exception car cela fonctionne comme le mécanisme des assertions : la première exception est relevée mais non capturée.

\subsection{Capturer une exception}

Pour vérifier que les deux exceptions sont levées, il est nécessaire de créer un bloc try{...}catch(){...} par test.
Chaque bloc try catch permet de gérer l'exception déclenchée dans le try et de la gérer grâce à l'instruction du bloc catch.
Dans la partie catch les variables n'ont pas de valeur car si on passe dans le bloc catch, cela signifie que l'instanciation de JaugeNaturel à échoué.  




\end{document}
